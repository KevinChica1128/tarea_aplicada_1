\documentclass[12pt]{beamer}
\usetheme{CambridgeUS}
\usepackage[utf8]{inputenc}
\usepackage[spanish]{babel}
\usepackage{amsmath}
\usepackage{amsfonts}
\usepackage{amssymb}
\usepackage{graphicx}
\author{Kevin Garcia - Alejandro Vargas}
\title{Modelo de regresión lineal múltiple}
%\setbeamercovered{transparent} 
%\setbeamertemplate{navigation symbols}{} 
%\logo{} 
%\institute{} 
%\date{} 
%\subject{} 
\begin{document}

\begin{frame}
\titlepage
\end{frame}

%\begin{frame}
%\tableofcontents
%\end{frame}

\begin{frame}
\frametitle{Análisis exploratorio de datos}
~\\ Para trabajar con la base de datos denominada 'cadata', generamos un número aleatorio con la ayuda del software R, el cuál nos arrojó el número 15529, por tanto nuestra base de datos final, quedo con las 9 variables (columnas) y con las filas desde la 15529 hasta la 16028.
\end{frame}

\end{document}